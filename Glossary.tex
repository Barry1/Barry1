% lacheck runs fine
% chktex runs fine
% latexindent --overwrite
%https://en.wikibooks.org/wiki/LaTeX/Glossary
%%%%%%%%%%%%%%%%%%%%%%%%%%%%%%%%%%%%%%%%%%%%%
%% in your preamble - hyperref before glossaries
% \usepackage[nomain,acronym,xindy,toc]{glossaries} % nomain, if you define glossaries in a file, and you use \include{INP-00-glossary}
% \makeglossaries
% \usepackage[xindy]{imakeidx}
% \makeindex
%% Load in your document by
% preferred \loadglsentries[main]{Glossary}
% or % lacheck runs fine
% chktex runs fine
% latexindent --overwrite
%https://en.wikibooks.org/wiki/LaTeX/Glossary
%%%%%%%%%%%%%%%%%%%%%%%%%%%%%%%%%%%%%%%%%%%%%
%% in your preamble - hyperref before glossaries
% \usepackage[nomain,acronym,xindy,toc]{glossaries} % nomain, if you define glossaries in a file, and you use \include{INP-00-glossary}
% \makeglossaries
% \usepackage[xindy]{imakeidx}
% \makeindex
%% Load in your document by
% preferred \loadglsentries[main]{Glossary}
% or % lacheck runs fine
% chktex runs fine
% latexindent --overwrite
%https://en.wikibooks.org/wiki/LaTeX/Glossary
%%%%%%%%%%%%%%%%%%%%%%%%%%%%%%%%%%%%%%%%%%%%%
%% in your preamble - hyperref before glossaries
% \usepackage[nomain,acronym,xindy,toc]{glossaries} % nomain, if you define glossaries in a file, and you use \include{INP-00-glossary}
% \makeglossaries
% \usepackage[xindy]{imakeidx}
% \makeindex
%% Load in your document by
% preferred \loadglsentries[main]{Glossary}
% or % lacheck runs fine
% chktex runs fine
% latexindent --overwrite
%https://en.wikibooks.org/wiki/LaTeX/Glossary
%%%%%%%%%%%%%%%%%%%%%%%%%%%%%%%%%%%%%%%%%%%%%
%% in your preamble - hyperref before glossaries
% \usepackage[nomain,acronym,xindy,toc]{glossaries} % nomain, if you define glossaries in a file, and you use \include{INP-00-glossary}
% \makeglossaries
% \usepackage[xindy]{imakeidx}
% \makeindex
%% Load in your document by
% preferred \loadglsentries[main]{Glossary}
% or \input{Glossary}
% BEFORE \begin{document}
% and print by
% \appendix
% \printindex
% \setglossarystyle{<newstyle>}
% \printglossary[title=List of Terms,toctitle=Terms and abbreviations]
% BEFORE \end{document}
%%%%%%%%%%%%%%%%%%%%%%%%%%%%%%%%%%%%%%%%%%%%%
\newglossary[slg]{symbols}{syi}{syg}{Symbolverzeichnis}%
\newglossaryentry{Functional claim}%
{%
	name=Functional claim,%
	description={\dots{} represents our value proposition to our markets and customers by addressing a specific market need through the delivery of defined functions and capabilities. It reflects our expertise and responsibility in providing solutions that meet customer requirements and create differentiated value}%
}
\newglossaryentry{Functional chain}%
{%
	name=Functional chain,%
	description={\dots{} is a sequence of interconnected functions that work  together to achieve a specific operation and functionality of a system}%
}
\newglossaryentry{Product}%
{%
	name=Product,%
	description={A \dots{} is sold based on a data sheet. %
		Research and Development (R\&D) is responsible for ensuring conformity with this data sheet. %
		Additionally, the entire process follows the product lifecycle management (PLM) framework. %
		Project execution excellence refers to the ability to deliver projects that consistently meet quality, %
		time and cost objectives through well-defined processes, advanced methods and tools and a highly skilled organization. %
		It ensures that performance contracts are executed efficiently, %
		minimizing risks while maximizing value for both the customer and the organization}%
}
\newglossaryentry{Solution}%#
{%
	name=Solution,%
	description={Solutions in the Scaled Agile Framework® (SAFe®) are services, products, or systems for the customer (external or internal). %
		In SAFe®, each new development value stream develops one or more solutions. \url{https://archive.ph/pbNug}}%
}
\newglossaryentry{System}%
{%
	name=System,%
	description={A \dots{} is an organized set of components, processes and technologies that work together to perform a specific function or achieve a defined outcome. %
		They are designed to integrate seamlessly, ensuring efficiency, reliability and adaptability to meet customer or operational needs. %
		In contrast to products, a system is sold through a performance contract. %
		The business entity with PnL responsibility for the system provides a systems engineering team for ensuring compliance with the contract. %
		This process is specific to each customer contract and should follow an implementation of customer projects framework}%
}
\newglossaryentry{System of Systems}%
{%
	name=System of Systems,%
	description={A \dots {} refers to a collection of independent systems that work together to achieve a higher-level capability or objective that no single system could accomplish alone. %
		These systems maintain their own operational independence while being integrated through coordination, communication and interoperability to deliver complex mission-critical solutions}%
}
\newglossaryentry{Value Proposition}%
{%
	name=Value Proposition,%
	description={\dots {} Value proposition refers to the unique benefits we offer to address customer needs and differentiate ourselves from competitors. %
		It highlights how our products or services solve problems and deliver value effectively}%
}
\newglossaryentry{sym:pi}{%
	type=symbols,%
	name={\ensuremath{\pi}},%
	sort=pi,%
	description={Kreiszahl, Verhältnis von Umfang zu Durchmesser eines Kreises}%
}
\newacronym{aia}{AIA}{Aerospace Industries Association of America}%
\newacronym{api}{API}{Application Programming Interface}%
\newacronym{asd}{ASD}{Aerospace, Security and Defence Industries Association of Europe}%
\newacronym{cpu}{CPU}{Central Processing Unit}%
\newacronym{cra}{CRA}{Cyber Resilience Act \url{https://www.bsi.bund.de/dok/cra-en} and \url{https://digital-strategy.ec.europa.eu/en/policies/cyber-resilience-act}}%
\newacronym{cwix}{CWIX}{Coalition Warrior Interoperability Exercise}%
\newacronym{edstart}{EDSTART}{European Defence Standardisation Refence System \url{https://edstar.eda.europa.eu/}}%
\newacronym{fuewes}{FüWES}{Führungs- und Waffeneinsatz-System}%
\newacronym{gpu}{GPU}{Graphics Processing Unit}%
\newacronym{ietd}{IETD}{Interaktive Elektronische Technische Dokumentation}%
\newacronym{ilasst}{ILASST}{Integriertes Leit- und Automationssystem Schiffstechnik}%
\newacronym{mod}{MoD}{Ministry of Defence}%
\newacronym{mukdo}{MUKdo}{Marine Unterstützungskommando}%
\newacronym{öag}{öAG}{öffentlicher Auftraggeber}%
\newacronym{padm}{PAD-M}{Process Associated Document --- Mandatory}%
\newacronym{pcp}{PCP}{Product Compliance Project}%
\newacronym{pnl}{PnL}{Profit and Loss}%
\newacronym{rohs}{RoHS}{Restriction of Hazardous Substances}%
\newacronym{safe}{SAFe}{Scaled Agile Framework}%
\newacronym{sagd}{SAGD}{Schadensabwehr- und Gefechtsdienst}%
\newacronym{sbom}{SBOM}{Software Bill of Materials}%
\newacronym{sdl}{SDL}{Secure Development Lifecycle see \href{https://www.vde-verlag.de/iec-normen/249036/iec-62443-4-1-2018.html}{IEC 62443-4-1:2018}}%
\newacronym{tara}{TARA}{Threat Analysis and Risk Assessment}%
\newacronym{tavz}{TAzV}{Technische Anlage zum Vertrag}%
\newacronym{usv}{USV}{Unterbrechungsfreie Stromversorgung}%
\newacronym{wtd}{WTD}{Wehrtechnische Dienststelle}%
\newacronym{zdv}{ZDv}{Zentrale Dienstvorschrift}%

% BEFORE \begin{document}
% and print by
% \appendix
% \printindex
% \setglossarystyle{<newstyle>}
% \printglossary[title=List of Terms,toctitle=Terms and abbreviations]
% BEFORE \end{document}
%%%%%%%%%%%%%%%%%%%%%%%%%%%%%%%%%%%%%%%%%%%%%
\newglossary[slg]{symbols}{syi}{syg}{Symbolverzeichnis}%
\newglossaryentry{Functional claim}%
{%
	name=Functional claim,%
	description={\dots{} represents our value proposition to our markets and customers by addressing a specific market need through the delivery of defined functions and capabilities. It reflects our expertise and responsibility in providing solutions that meet customer requirements and create differentiated value}%
}
\newglossaryentry{Functional chain}%
{%
	name=Functional chain,%
	description={\dots{} is a sequence of interconnected functions that work  together to achieve a specific operation and functionality of a system}%
}
\newglossaryentry{Product}%
{%
	name=Product,%
	description={A \dots{} is sold based on a data sheet. %
		Research and Development (R\&D) is responsible for ensuring conformity with this data sheet. %
		Additionally, the entire process follows the product lifecycle management (PLM) framework. %
		Project execution excellence refers to the ability to deliver projects that consistently meet quality, %
		time and cost objectives through well-defined processes, advanced methods and tools and a highly skilled organization. %
		It ensures that performance contracts are executed efficiently, %
		minimizing risks while maximizing value for both the customer and the organization}%
}
\newglossaryentry{Solution}%#
{%
	name=Solution,%
	description={Solutions in the Scaled Agile Framework® (SAFe®) are services, products, or systems for the customer (external or internal). %
		In SAFe®, each new development value stream develops one or more solutions. \url{https://archive.ph/pbNug}}%
}
\newglossaryentry{System}%
{%
	name=System,%
	description={A \dots{} is an organized set of components, processes and technologies that work together to perform a specific function or achieve a defined outcome. %
		They are designed to integrate seamlessly, ensuring efficiency, reliability and adaptability to meet customer or operational needs. %
		In contrast to products, a system is sold through a performance contract. %
		The business entity with PnL responsibility for the system provides a systems engineering team for ensuring compliance with the contract. %
		This process is specific to each customer contract and should follow an implementation of customer projects framework}%
}
\newglossaryentry{System of Systems}%
{%
	name=System of Systems,%
	description={A \dots {} refers to a collection of independent systems that work together to achieve a higher-level capability or objective that no single system could accomplish alone. %
		These systems maintain their own operational independence while being integrated through coordination, communication and interoperability to deliver complex mission-critical solutions}%
}
\newglossaryentry{Value Proposition}%
{%
	name=Value Proposition,%
	description={\dots {} Value proposition refers to the unique benefits we offer to address customer needs and differentiate ourselves from competitors. %
		It highlights how our products or services solve problems and deliver value effectively}%
}
\newglossaryentry{sym:pi}{%
	type=symbols,%
	name={\ensuremath{\pi}},%
	sort=pi,%
	description={Kreiszahl, Verhältnis von Umfang zu Durchmesser eines Kreises}%
}
\newacronym{aia}{AIA}{Aerospace Industries Association of America}%
\newacronym{api}{API}{Application Programming Interface}%
\newacronym{asd}{ASD}{Aerospace, Security and Defence Industries Association of Europe}%
\newacronym{cpu}{CPU}{Central Processing Unit}%
\newacronym{cra}{CRA}{Cyber Resilience Act \url{https://www.bsi.bund.de/dok/cra-en} and \url{https://digital-strategy.ec.europa.eu/en/policies/cyber-resilience-act}}%
\newacronym{cwix}{CWIX}{Coalition Warrior Interoperability Exercise}%
\newacronym{edstart}{EDSTART}{European Defence Standardisation Refence System \url{https://edstar.eda.europa.eu/}}%
\newacronym{fuewes}{FüWES}{Führungs- und Waffeneinsatz-System}%
\newacronym{gpu}{GPU}{Graphics Processing Unit}%
\newacronym{ietd}{IETD}{Interaktive Elektronische Technische Dokumentation}%
\newacronym{ilasst}{ILASST}{Integriertes Leit- und Automationssystem Schiffstechnik}%
\newacronym{mod}{MoD}{Ministry of Defence}%
\newacronym{mukdo}{MUKdo}{Marine Unterstützungskommando}%
\newacronym{öag}{öAG}{öffentlicher Auftraggeber}%
\newacronym{padm}{PAD-M}{Process Associated Document --- Mandatory}%
\newacronym{pcp}{PCP}{Product Compliance Project}%
\newacronym{pnl}{PnL}{Profit and Loss}%
\newacronym{rohs}{RoHS}{Restriction of Hazardous Substances}%
\newacronym{safe}{SAFe}{Scaled Agile Framework}%
\newacronym{sagd}{SAGD}{Schadensabwehr- und Gefechtsdienst}%
\newacronym{sbom}{SBOM}{Software Bill of Materials}%
\newacronym{sdl}{SDL}{Secure Development Lifecycle see \href{https://www.vde-verlag.de/iec-normen/249036/iec-62443-4-1-2018.html}{IEC 62443-4-1:2018}}%
\newacronym{tara}{TARA}{Threat Analysis and Risk Assessment}%
\newacronym{tavz}{TAzV}{Technische Anlage zum Vertrag}%
\newacronym{usv}{USV}{Unterbrechungsfreie Stromversorgung}%
\newacronym{wtd}{WTD}{Wehrtechnische Dienststelle}%
\newacronym{zdv}{ZDv}{Zentrale Dienstvorschrift}%

% BEFORE \begin{document}
% and print by
% \appendix
% \printindex
% \setglossarystyle{<newstyle>}
% \printglossary[title=List of Terms,toctitle=Terms and abbreviations]
% BEFORE \end{document}
%%%%%%%%%%%%%%%%%%%%%%%%%%%%%%%%%%%%%%%%%%%%%
\newglossary[slg]{symbols}{syi}{syg}{Symbolverzeichnis}%
\newglossaryentry{Functional claim}%
{%
	name=Functional claim,%
	description={\dots{} represents our value proposition to our markets and customers by addressing a specific market need through the delivery of defined functions and capabilities. It reflects our expertise and responsibility in providing solutions that meet customer requirements and create differentiated value}%
}
\newglossaryentry{Functional chain}%
{%
	name=Functional chain,%
	description={\dots{} is a sequence of interconnected functions that work  together to achieve a specific operation and functionality of a system}%
}
\newglossaryentry{Product}%
{%
	name=Product,%
	description={A \dots{} is sold based on a data sheet. %
		Research and Development (R\&D) is responsible for ensuring conformity with this data sheet. %
		Additionally, the entire process follows the product lifecycle management (PLM) framework. %
		Project execution excellence refers to the ability to deliver projects that consistently meet quality, %
		time and cost objectives through well-defined processes, advanced methods and tools and a highly skilled organization. %
		It ensures that performance contracts are executed efficiently, %
		minimizing risks while maximizing value for both the customer and the organization}%
}
\newglossaryentry{Solution}%#
{%
	name=Solution,%
	description={Solutions in the Scaled Agile Framework® (SAFe®) are services, products, or systems for the customer (external or internal). %
		In SAFe®, each new development value stream develops one or more solutions. \url{https://archive.ph/pbNug}}%
}
\newglossaryentry{System}%
{%
	name=System,%
	description={A \dots{} is an organized set of components, processes and technologies that work together to perform a specific function or achieve a defined outcome. %
		They are designed to integrate seamlessly, ensuring efficiency, reliability and adaptability to meet customer or operational needs. %
		In contrast to products, a system is sold through a performance contract. %
		The business entity with PnL responsibility for the system provides a systems engineering team for ensuring compliance with the contract. %
		This process is specific to each customer contract and should follow an implementation of customer projects framework}%
}
\newglossaryentry{System of Systems}%
{%
	name=System of Systems,%
	description={A \dots {} refers to a collection of independent systems that work together to achieve a higher-level capability or objective that no single system could accomplish alone. %
		These systems maintain their own operational independence while being integrated through coordination, communication and interoperability to deliver complex mission-critical solutions}%
}
\newglossaryentry{Value Proposition}%
{%
	name=Value Proposition,%
	description={\dots {} Value proposition refers to the unique benefits we offer to address customer needs and differentiate ourselves from competitors. %
		It highlights how our products or services solve problems and deliver value effectively}%
}
\newglossaryentry{sym:pi}{%
	type=symbols,%
	name={\ensuremath{\pi}},%
	sort=pi,%
	description={Kreiszahl, Verhältnis von Umfang zu Durchmesser eines Kreises}%
}
\newacronym{aia}{AIA}{Aerospace Industries Association of America}%
\newacronym{api}{API}{Application Programming Interface}%
\newacronym{asd}{ASD}{Aerospace, Security and Defence Industries Association of Europe}%
\newacronym{cpu}{CPU}{Central Processing Unit}%
\newacronym{cra}{CRA}{Cyber Resilience Act \url{https://www.bsi.bund.de/dok/cra-en} and \url{https://digital-strategy.ec.europa.eu/en/policies/cyber-resilience-act}}%
\newacronym{cwix}{CWIX}{Coalition Warrior Interoperability Exercise}%
\newacronym{edstart}{EDSTART}{European Defence Standardisation Refence System \url{https://edstar.eda.europa.eu/}}%
\newacronym{fuewes}{FüWES}{Führungs- und Waffeneinsatz-System}%
\newacronym{gpu}{GPU}{Graphics Processing Unit}%
\newacronym{ietd}{IETD}{Interaktive Elektronische Technische Dokumentation}%
\newacronym{ilasst}{ILASST}{Integriertes Leit- und Automationssystem Schiffstechnik}%
\newacronym{mod}{MoD}{Ministry of Defence}%
\newacronym{mukdo}{MUKdo}{Marine Unterstützungskommando}%
\newacronym{öag}{öAG}{öffentlicher Auftraggeber}%
\newacronym{padm}{PAD-M}{Process Associated Document --- Mandatory}%
\newacronym{pcp}{PCP}{Product Compliance Project}%
\newacronym{pnl}{PnL}{Profit and Loss}%
\newacronym{rohs}{RoHS}{Restriction of Hazardous Substances}%
\newacronym{safe}{SAFe}{Scaled Agile Framework}%
\newacronym{sagd}{SAGD}{Schadensabwehr- und Gefechtsdienst}%
\newacronym{sbom}{SBOM}{Software Bill of Materials}%
\newacronym{sdl}{SDL}{Secure Development Lifecycle see \href{https://www.vde-verlag.de/iec-normen/249036/iec-62443-4-1-2018.html}{IEC 62443-4-1:2018}}%
\newacronym{tara}{TARA}{Threat Analysis and Risk Assessment}%
\newacronym{tavz}{TAzV}{Technische Anlage zum Vertrag}%
\newacronym{usv}{USV}{Unterbrechungsfreie Stromversorgung}%
\newacronym{wtd}{WTD}{Wehrtechnische Dienststelle}%
\newacronym{zdv}{ZDv}{Zentrale Dienstvorschrift}%

% BEFORE \begin{document}
% and print by
% \appendix
% \printindex
% \setglossarystyle{<newstyle>}
% \printglossary[title=List of Terms,toctitle=Terms and abbreviations]
% BEFORE \end{document}
%%%%%%%%%%%%%%%%%%%%%%%%%%%%%%%%%%%%%%%%%%%%%
\newglossary[slg]{symbols}{syi}{syg}{Symbolverzeichnis}%
\newglossaryentry{a4ESSOR S.A.S}%
{%
	name=a4ESSOR S.A.S,%
	description={Under the umbrella of ESSOR and with OCCAR as the program management organization, %
	Spain, Italy, France, Finland, Poland and, since 2020, Germany have set themselves the goal of achieving this. %
	The leading communications companies of these participating states formed an alliance for ESSOR, %
	the joint venture, a4ESSOR S.A.S., which is in charge of managing the industrial consortium composed of the respective national champions}%
}
\newglossaryentry{CWIX}%
{%
	name=CWIX,%
	description={CWIX (Coalition Warrior Interoperability eXploration, eXperimentation, eXamination eXercise) is NATO's premier annual event 
		for testing and enhancing the interoperability of command, control, and information systems among Alliance and partner nations.
		Held at the Joint Force Training Centre in Bydgoszcz, Poland, CWIX-25 brought together 3,000+ experts from 42 nations
		to validate secure digital communication, space, and land capabilities}%
}
\newglossaryentry{Digital Twin}%
{%
	name=Digital Twin,%
	description={A digital twin is a digital representation of a product, process, or system either in operation or in development.
	    When in operation, it reflects the asset’s current condition and includes relevant historical data;
		digital twins are used to evaluate an asset’s current state and, more importantly, to predict future behavior, refine control systems, or optimize operations.
		During development, the digital twin acts as a model of a to-be-built product, process, or system that facilitates development, testing, and validation.
		\href{https://www.mathworks.com/discovery/digital-twin.html}{Source}}%
}
\newglossaryentry{ESSOR}%
{%
	name=ESSOR,%
	description={The European Secure Software Defined Radio (ESSOR) project is one of Europe's most successful collaborative defense projects.
		It ensures interoperability across nations through innovative, future-proof technology.
		The main scope of this project is to provide a waveform suite to enable secure collaborative communications between European forces in joint and combined missions.
		At the core of the project is a shared set of rules and a shared methodology on how to ensure a common approach to achieve portable waveforms as well as verifying and validating waveform interoperability.
		\newline{}%
		Currently, the ESSOR high data rate waveform (HDRWF) Operational Capability 1 (OC1) is being implemented in the SOVERON D radio from Rohde \& Schwarz.
		Within the framework of the European Defence Investment Programme (EDIDP), the roadmap ahead sees the development of a jamming-robust tactical waveform (ESSOR Narrowband Waveform = ENBWF);
		the development of a jamming-robust UHF flight traffic waveform for ground to air to ground communications (ESSOR 3D Waveform = E3DWF);
		the definition and setup of an ESSOR test and validation center (ESSOR and validating ESSOR In-Service Support Framework);
		and the conception of a possible future EU satellite communications waveform.
		\href{https://www.rohde-schwarz.com/de/loesungen/aerospace-and-defense/land/tactical-communications/essor/essor_258052.html}{Source}=\url{https://archive.today/qNVCT}
	}%
}
\newglossaryentry{Federated Mission Networking}%
{%
	name=Federated Mission Networking,%
	description={Das Federated Mission Networking ist die Art wie sich die einzelnen Mitglieder, in der Fachsprache Affiliates genannt,
		zusammenfinden, um einen gemeinsamen Stand an Interoperabilität zu erreichen.
		In der NATO bringen die Mitglieder ganz unterschiedliche Voraussetzungen mit,
		sowohl in der Technik als auch im Bereich der Einsatzgrundsätze und der Ausbildung.
		In einer Föderation werden die eigenen Netzwerke und Systeme beibehalten und unter dem Dach des FMN aufeinander abgestimmt.
		So kann die NATO schneller auf Krisen reagieren und gemeinsam agieren, 
		ohne dass die jeweilige nationale Selbstständigkeit aufgegeben werden muss.
		Die NATO spricht deshalb in diesem Zusammenhang von Day Zero Interoperability,
		in Deutschland wird das häufig mit Kaltstartfähigkeit übersetzt.
		\href{https://www.bundeswehr.de/de/organisation/cyber-und-informationsraum/aktuelles/fmn-grundlage-einsaetze-uebungen-5421864}{Weitere Details}.
		\newline{}
		Die Bundeswehr setzt diese multinational vereinbarten Standards unter anderem im Projekt German Mission Network (GMN) um.
		Das GMN ist also eine Möglichkeit, die multinationalen Vorgaben des FMN für ein Land wie Deutschland zu verwirklichen.
		\href{https://www.bundeswehr.de/de/organisation/cyber-und-informationsraum/aktuelles/nato-uebung-cwix-interview-5956132}{Source}=\url{https://archive.today/rJwjx}
		}%
}
\newglossaryentry{Functional claim}%
{%
	name=Functional claim,%
	description={\dots{} represents our value proposition to our markets and customers by addressing a specific market need through the delivery of defined functions and capabilities. It reflects our expertise and responsibility in providing solutions that meet customer requirements and create differentiated value}%
}
\newglossaryentry{Functional chain}%
{%
	name=Functional chain,%
	description={\dots{} is a sequence of interconnected functions that work  together to achieve a specific operation and functionality of a system}%
}
\newglossaryentry{Last-Time Buy}%
{%
	name=Last-Time Buy,%
	description={\dots{} is the final opportunity to purchase a discontinued item, usually with a deadline, to stock up for future needs}%
}
\newglossaryentry{Product}%
{%
	name=Product,%
	description={A \dots{} is sold based on a data sheet. %
		Research and Development (R\&D) is responsible for ensuring conformity with this data sheet. %
		Additionally, the entire process follows the product lifecycle management (PLM) framework. %
		Project execution excellence refers to the ability to deliver projects that consistently meet quality, %
		time and cost objectives through well-defined processes, advanced methods and tools and a highly skilled organization. %
		It ensures that performance contracts are executed efficiently, %
		minimizing risks while maximizing value for both the customer and the organization}%
}
\newglossaryentry{Solution}%#
{%
	name=Solution,%
	description={Solutions in the Scaled Agile Framework® (SAFe®) are services, products, or systems for the customer (external or internal). %
		In SAFe®, each new development value stream develops one or more solutions. %
		Source \url{https://www.planforge.io/en/knowledge/glossary/solution-safe} or archived \url{https://archive.today/pbNug}}%
}
\newglossaryentry{Subcription}%#
{%
	name=Subcription,%
	description={Subscription oder Subscription Economy ist ein Geschäftsmodell, %
	    das wie ein Abonnement funktioniert und in mehr und mehr Branchen genutzt wird, %
		vor allem aber im Bereich Software und Cloud Services. %
		Der Begriff ist abgeleitet von den lateinischen Wörtern „sub“ (unter) und „scribere“ (schreiben), %
		bedeutet also „unterschreiben“ oder „eine Unterschrift“ leisten. Konkret geht es darum, %
		regelmäßig einen festgelegten Geldbetrag für die Nutzung von Produkten oder Dienstleistungen zu zahlen, %
		die von einem Lieferanten bereitgestellt werden, statt diese zu kaufen und zu besitzen. %
		Source \url{https://www.cloudcomputing-insider.de/was-bedeutet-subscription-a-769ad13c279bfeaed664c3edc69778a7} or archived \url{https://archive.today/u0qNU}}%
}
\newglossaryentry{System}%
{%
	name=System,%
	description={A \dots{} is an organized set of components, processes and technologies that work together to perform a specific function or achieve a defined outcome. %
		They are designed to integrate seamlessly, ensuring efficiency, reliability and adaptability to meet customer or operational needs. %
		In contrast to products, a system is sold through a performance contract. %
		The business entity with PnL responsibility for the system provides a systems engineering team for ensuring compliance with the contract. %
		This process is specific to each customer contract and should follow an implementation of customer projects framework}%
}
\newglossaryentry{System of Systems}%
{%
	name=System of Systems,%
	description={A \dots {} refers to a collection of independent systems that work together to achieve a higher-level capability or objective that no single system could accomplish alone. %
		These systems maintain their own operational independence while being integrated through coordination, communication and interoperability to deliver complex mission-critical solutions}%
}
\newglossaryentry{Tuckman's Team Development Model}%
{%
	name=Tuckman's Team Development Model,%
	description={Bruce Tuckman, a psychology professor, identified four stages of development --- forming, storming, norming and performing ---
	    that every team experiences, and suggested that all teams go through a relatively unproductive initial stage before becoming a self-reliant unit.
		The ‘team growth model’ also suggests that unless the issues of processes and feelings have been satisfactorily addressed,
		it is unlikely that the team will reach the most productive final stage.
		\href{https://www.uwindsor.ca/ctl/sites/uwindsor.ca.ctl/files/bruce_tuckmans_stages_of_team_development.pdf}{Source}=\url{https://archive.today/A1hEi}}%
}
\newglossaryentry{Value Proposition}%
{%
	name=Value Proposition,%
	description={\dots {} Value proposition refers to the unique benefits we offer to address customer needs and differentiate ourselves from competitors. %
		It highlights how our products or services solve problems and deliver value effectively}%
}
\newglossaryentry{sym:pi}{%
	type=symbols,%
	name={\ensuremath{\pi}},%
	sort=pi,%
	description={Kreiszahl, Verhältnis von Umfang zu Durchmesser eines Kreises}%
}
\newacronym{aaw}{AAW}{Anti Air Warfare}%
\newacronym{aia}{AIA}{Aerospace Industries Association of America}%
\newacronym{act}{ACT}{Allied Command Transformation \href{http://act.nato.int/}{NATO}}%
\newacronym{api}{API}{Application Programming Interface}%
\newacronym{asd}{ASD}{Aerospace, Security and Defence Industries Association of Europe}%
\newacronym{asuw}{ASuW}{Anti Surface Warfare}%
\newacronym{asw}{ASW}{Anti Submarine Warfare}%
\newacronym{aura}{AURA}{Advanced Unified Radio Architecture}%
\newacronym{bfn}{BFN}{Beam Forming Network}%
\newacronym{capex}{CAPEX}{Capital Expenditures}%
\newacronym{ccb}{CCB}{Change Control Board}%
\newacronym{ccv}{CCV}{Collaborative Combat Vessel}%
\newacronym{cdr}{CDR}{Critical Design Review}%
\newacronym{cots}{COTS}{commercial off-the-shelf}%
\newacronym{cpu}{CPU}{Central Processing Unit}%
\newacronym{cra}{CRA}{Cyber Resilience Act \url{https://www.bsi.bund.de/dok/cra-en} and \url{https://digital-strategy.ec.europa.eu/en/policies/cyber-resilience-act}}%
\newacronym{cwix}{CWIX}{Coalition Warrior Interoperability Exercise}%
\newacronym{dlep}{DLEP}{Dynamic Link Exchange Protocol \href{https://datatracker.ietf.org/doc/rfc8175/}{RFC 8175}}%
\newacronym{dod}{DoD}{Definition of Done}%
\newacronym{e3dwf}{E3DWF}{ESSOR Air-Ground-Air Communications Waveform}%
\newacronym{ec}{EC}{European Commission}%
\newacronym{eccm}{ECCM}{electronic counter-countermeasures}%
\newacronym{ecdis}{ECDIS}{electronic chart display and information systems}%
\newacronym{ecmp}{ECMP}{electronic components management plan}%
\newacronym{edc}{EDC}{Effective Date of Contract}%
\newacronym{edstart}{EDSTART}{European Defence Standardisation Refence System \href{https://edstar.eda.europa.eu/}{EDA Homepage}}%
\newacronym{ehdrwf}{EHDRWF}{ESSOR High data rate Waveform}%
\newacronym{eme}{EME}{ElectroMagnetic Environment}%
\newacronym{enbwf}{ENBWF}{ESSOR Narrowband Waveform}%
\newacronym{eol}{EOL}{end of life}%
\newacronym{eop}{EOP}{end of production}%
\newacronym{eos}{EOS}{end of support}%
\newacronym{epm}{EPM}{Electronic Protective Measures}
\newacronym{esa}{ESA}{European Space Agency}%
\newacronym{esatwf}{ESATWF}{ESSOR Satellite Communications Waveform}%
\newacronym{esd}{ESD}{electrostatic discharge}%
\newacronym{fam}{FAM}{Frequency Antenna Management}%
\newacronym{fat}{FAT}{Factory Acceptance Test}%
\newacronym{fmi}{FMI}{Functional Mockup Interface}%
\newacronym{fmn}{FMN}{Federated Mission Networking}%
\newacronym{fuewes}{FüWES}{Führungs- und Waffeneinsatz-System}%
\newacronym{gmdss}{GMDSS}{Global Maritime Distress and Safety System}%
\newacronym{gmn}{GMN}{German Mission Network}%
\newacronym{gpu}{GPU}{Graphics Processing Unit}%
\newacronym{hat}{HAT}{Harbour Acceptance Test}%
\newacronym{ietd}{IETD}{Interaktive Elektronische Technische Dokumentation}%
\newacronym{iiom}{IIOM}{International Institute of Obsolescence Management}%
\newacronym{ilasst}{ILASST}{Integriertes Leit- und Automationssystem Schiffstechnik}%
\newacronym{imus}{IMUS}{Integriertes Message Handling und Steuerungssystem (Vorgänger KMS)}%
\newacronym{inbs}{INBS}{Integrated Navigation and Bridge System}%
\newacronym{ipm}{IPM}{Intellectual Propoerty Management}%
\newacronym{ipm2}{IPM}{Internet Protocol Management}%
\newacronym{ipm3}{IPM}{Internal Programm Manager}%
\newacronym{ipms}{IPMS}{Integrated Platform Management System}%
\newacronym{ipr}{IPR}{intellectual property rights}%
\newacronym{jitc}{JITC}{Joint Interopearability Test Command}%
\newacronym{kms}{KMS}{Kommunikations-Management-System}%
\newacronym{lbif}{LBIF}{Land based integration facility}%
\newacronym{lnb}{LNB}{Life of Need Buy}%
\newacronym{ltb}{LTB}{Last-Time Buy}%
\newacronym{mgs}{MGS}{Material-Gruppen-Strategie}%
\newacronym{mhc}{MHC}{Mine Hunting Capability}%
\newacronym{mmr}{MMR}{Military Minimum Requirement}%
\newacronym{mne}{MNE}{Maritime Network Evolution \href{https://archive.today/C8Iqp}{Airbus and Rohde \& Schwarz boost Royal Navy connectivity}}%
\newacronym{mod}{MoD}{Ministry of Defence}%
\newacronym{mou}{MoU}{Memorandum of Understanding}%
\newacronym{msc}{MSC}{MainSubContractor}%
\newacronym{mukdo}{MUKdo}{Marine Unterstützungskommando}%
\newacronym{nte}{NTE}{Native Test Equipment}%
\newacronym{öag}{öAG}{öffentlicher Auftraggeber}%
\newacronym{ocm}{OCM}{original component manufacturer}%
\newacronym{oem}{OEM}{original equipment manufacturer}%
\newacronym{omp}{OMP}{obsolescence management plan}%
\newacronym{opex_f}{OpEx}{Operational Expenditures}%
\newacronym{opex_n}{OpEx}{Operational Experimentation \href{https://archive.ph/QYJtq}{test}}%#selection-2159.17-2159.44:~:text=Operational%20Experimentation
\newacronym{osp}{OSP}{Open Simulation Platform \url{https://opensimulationplatform.com/}}%
\newacronym{padm}{PAD-M}{Process Associated Document --- Mandatory}%
\newacronym{pcn}{PCN}{product change notice}%
\newacronym{pcp}{PCP}{Product Compliance Project}%
\newacronym{pdn}{PDN}{product discontinuance notice}%
\newacronym{pdr}{PDR}{Preliminary Design Review}%
\newacronym{pms}{PMS}{Platform Management System}%
\newacronym{pnl}{PnL}{Profit and Loss}%
\newacronym{psa}{psa}{please see above}%
\newacronym{psb}{psb}{please see below}%
\newacronym{psirt}{PSIRT}{Product Security Incident Response Team}%
\newacronym{ptp}{PtP}{Path to Progress}%
\newacronym{rar}{RAR}{Radio Aware Routing}%
\newacronym{rcg}{RCG}{Regulatory Compliance Group}%
\newacronym{reach}{REACH}{registration, evaluation, authorization and restriction of chemicals}%
\newacronym{red}{RED}{Radio Equipment Directive \href{https://single-market-economy.ec.europa.eu/sectors/electrical-and-electronic-engineering-industries-eei/radio-equipment-directive-red_en}{Richtlinie 2014/53/EU}}%
\newacronym{rhib}{RHIB}{Rigid Hull Inflatable Boat \href{https://www.bundeswehr.de/de/ausruestung-technik-bundeswehr/seesysteme-bundeswehr/festrumpfschlauchboot-buster}{Beispiel Buster auf F125}}%
\newacronym{rfof}{RFoF}{Radio Frequency over Fibre}%
\newacronym{rohs}{RoHS}{Restriction of Hazardous Substances}%
\newacronym{safe}{SAFe}{Scaled Agile Framework}%
\newacronym{sagd}{SAGD}{Schadensabwehr- und Gefechtsdienst}%
\newacronym{sat}{SAT}{Sea Acceptance Test}%
\newacronym{saturn}{SATURN}{Second Generation Anti-Jam Tactical UHF Radio for NATO}
\newacronym{sbom}{SBOM}{Software Bill of Materials}%
\newacronym{sdl}{SDL}{Secure Development Lifecycle see \href{https://www.vde-verlag.de/iec-normen/249036/iec-62443-4-1-2018.html}{IEC 62443-4-1:2018}}%
\newacronym{sme1}{SME}{Small and medium entities}%
\newacronym{sme2}{SME}{subject matter experts}%
\newacronym{swap}{SWaP}{Space, Weight and Power}%
\newacronym{tara}{TARA}{Threat Analysis and Risk Assessment, central part of ISO/SAE 21434-Norm}%
\newacronym{tavz}{TAzV}{Technische Anlage zum Vertrag}%
\newacronym{tlb}{TLB}{Technisch logistische Betreuung}%
\newacronym{tlp}{TLP}{Traffic Light Protocol \href{https://www.first.org/tlp/}{Standard}}%
\newacronym{tot}{ToT}{Transfer of Technology}%
\newacronym{ttp}{TTP}{Tactics, Techniques and Procedures}%
\newacronym{uad}{UAD}{User Access Device}%
\newacronym{usv_e}{USV}{Unterbrechungsfreie Stromversorgung}%
\newacronym{usv_n}{USV}{Unmanned Surface Vehicle}%
\newacronym{wtd}{WTD}{Wehrtechnische Dienststelle}%
\newacronym{zdv}{ZDv}{Zentrale Dienstvorschrift}%
