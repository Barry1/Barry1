%https://en.wikibooks.org/wiki/LaTeX/Glossary
%%%%%%%%%%%%%%%%%%%%%%%%%%%%%%%%%%%%%%%%%%%%%
%% in your preamble - hyperref before glossaries
% \usepackage[nomain,acronym,xindy,toc]{glossaries} % nomain, if you define glossaries in a file, and you use \include{INP-00-glossary}
% \makeglossaries
% \usepackage[xindy]{imakeidx}
% \makeindex
%% Load in your document by
% preferred \loadglsentries[main]{Glossary}
% or % lacheck runs fine
% chktex runs fine
% latexindent --overwrite
%https://en.wikibooks.org/wiki/LaTeX/Glossary
%%%%%%%%%%%%%%%%%%%%%%%%%%%%%%%%%%%%%%%%%%%%%
%% in your preamble - hyperref before glossaries
% \usepackage[nomain,acronym,xindy,toc]{glossaries} % nomain, if you define glossaries in a file, and you use \include{INP-00-glossary}
% \makeglossaries
% \usepackage[xindy]{imakeidx}
% \makeindex
%% Load in your document by
% preferred \loadglsentries[main]{Glossary}
% or % lacheck runs fine
% chktex runs fine
% latexindent --overwrite
%https://en.wikibooks.org/wiki/LaTeX/Glossary
%%%%%%%%%%%%%%%%%%%%%%%%%%%%%%%%%%%%%%%%%%%%%
%% in your preamble - hyperref before glossaries
% \usepackage[nomain,acronym,xindy,toc]{glossaries} % nomain, if you define glossaries in a file, and you use \include{INP-00-glossary}
% \makeglossaries
% \usepackage[xindy]{imakeidx}
% \makeindex
%% Load in your document by
% preferred \loadglsentries[main]{Glossary}
% or % lacheck runs fine
% chktex runs fine
% latexindent --overwrite
%https://en.wikibooks.org/wiki/LaTeX/Glossary
%%%%%%%%%%%%%%%%%%%%%%%%%%%%%%%%%%%%%%%%%%%%%
%% in your preamble - hyperref before glossaries
% \usepackage[nomain,acronym,xindy,toc]{glossaries} % nomain, if you define glossaries in a file, and you use \include{INP-00-glossary}
% \makeglossaries
% \usepackage[xindy]{imakeidx}
% \makeindex
%% Load in your document by
% preferred \loadglsentries[main]{Glossary}
% or \input{Glossary}
% BEFORE \begin{document}
% and print by
% \appendix
% \printindex
% \setglossarystyle{<newstyle>}
% \printglossary[title=List of Terms,toctitle=Terms and abbreviations]
% BEFORE \end{document}
%%%%%%%%%%%%%%%%%%%%%%%%%%%%%%%%%%%%%%%%%%%%%
\newglossary[slg]{symbols}{syi}{syg}{Symbolverzeichnis}%
\newglossaryentry{Functional claim}%
{%
	name=Functional claim,%
	description={\dots{} represents our value proposition to our markets and customers by addressing a specific market need through the delivery of defined functions and capabilities. It reflects our expertise and responsibility in providing solutions that meet customer requirements and create differentiated value}%
}
\newglossaryentry{Functional chain}%
{%
	name=Functional chain,%
	description={\dots{} is a sequence of interconnected functions that work  together to achieve a specific operation and functionality of a system}%
}
\newglossaryentry{Product}%
{%
	name=Product,%
	description={A \dots{} is sold based on a data sheet. %
		Research and Development (R\&D) is responsible for ensuring conformity with this data sheet. %
		Additionally, the entire process follows the product lifecycle management (PLM) framework. %
		Project execution excellence refers to the ability to deliver projects that consistently meet quality, %
		time and cost objectives through well-defined processes, advanced methods and tools and a highly skilled organization. %
		It ensures that performance contracts are executed efficiently, %
		minimizing risks while maximizing value for both the customer and the organization}%
}
\newglossaryentry{Solution}%#
{%
	name=Solution,%
	description={Solutions in the Scaled Agile Framework® (SAFe®) are services, products, or systems for the customer (external or internal). %
		In SAFe®, each new development value stream develops one or more solutions. \url{https://archive.ph/pbNug}}%
}
\newglossaryentry{System}%
{%
	name=System,%
	description={A \dots{} is an organized set of components, processes and technologies that work together to perform a specific function or achieve a defined outcome. %
		They are designed to integrate seamlessly, ensuring efficiency, reliability and adaptability to meet customer or operational needs. %
		In contrast to products, a system is sold through a performance contract. %
		The business entity with PnL responsibility for the system provides a systems engineering team for ensuring compliance with the contract. %
		This process is specific to each customer contract and should follow an implementation of customer projects framework}%
}
\newglossaryentry{System of Systems}%
{%
	name=System of Systems,%
	description={A \dots {} refers to a collection of independent systems that work together to achieve a higher-level capability or objective that no single system could accomplish alone. %
		These systems maintain their own operational independence while being integrated through coordination, communication and interoperability to deliver complex mission-critical solutions}%
}
\newglossaryentry{Value Proposition}%
{%
	name=Value Proposition,%
	description={\dots {} Value proposition refers to the unique benefits we offer to address customer needs and differentiate ourselves from competitors. %
		It highlights how our products or services solve problems and deliver value effectively}%
}
\newglossaryentry{sym:pi}{%
	type=symbols,%
	name={\ensuremath{\pi}},%
	sort=pi,%
	description={Kreiszahl, Verhältnis von Umfang zu Durchmesser eines Kreises}%
}
\newacronym{aia}{AIA}{Aerospace Industries Association of America}%
\newacronym{api}{API}{Application Programming Interface}%
\newacronym{asd}{ASD}{Aerospace, Security and Defence Industries Association of Europe}%
\newacronym{cpu}{CPU}{Central Processing Unit}%
\newacronym{cra}{CRA}{Cyber Resilience Act \url{https://www.bsi.bund.de/dok/cra-en} and \url{https://digital-strategy.ec.europa.eu/en/policies/cyber-resilience-act}}%
\newacronym{cwix}{CWIX}{Coalition Warrior Interoperability Exercise}%
\newacronym{edstart}{EDSTART}{European Defence Standardisation Refence System \url{https://edstar.eda.europa.eu/}}%
\newacronym{fuewes}{FüWES}{Führungs- und Waffeneinsatz-System}%
\newacronym{gpu}{GPU}{Graphics Processing Unit}%
\newacronym{ietd}{IETD}{Interaktive Elektronische Technische Dokumentation}%
\newacronym{ilasst}{ILASST}{Integriertes Leit- und Automationssystem Schiffstechnik}%
\newacronym{mod}{MoD}{Ministry of Defence}%
\newacronym{mukdo}{MUKdo}{Marine Unterstützungskommando}%
\newacronym{öag}{öAG}{öffentlicher Auftraggeber}%
\newacronym{padm}{PAD-M}{Process Associated Document --- Mandatory}%
\newacronym{pcp}{PCP}{Product Compliance Project}%
\newacronym{pnl}{PnL}{Profit and Loss}%
\newacronym{rohs}{RoHS}{Restriction of Hazardous Substances}%
\newacronym{safe}{SAFe}{Scaled Agile Framework}%
\newacronym{sagd}{SAGD}{Schadensabwehr- und Gefechtsdienst}%
\newacronym{sbom}{SBOM}{Software Bill of Materials}%
\newacronym{sdl}{SDL}{Secure Development Lifecycle see \href{https://www.vde-verlag.de/iec-normen/249036/iec-62443-4-1-2018.html}{IEC 62443-4-1:2018}}%
\newacronym{tara}{TARA}{Threat Analysis and Risk Assessment}%
\newacronym{tavz}{TAzV}{Technische Anlage zum Vertrag}%
\newacronym{usv}{USV}{Unterbrechungsfreie Stromversorgung}%
\newacronym{wtd}{WTD}{Wehrtechnische Dienststelle}%
\newacronym{zdv}{ZDv}{Zentrale Dienstvorschrift}%

% BEFORE \begin{document}
% and print by
% \appendix
% \printindex
% \setglossarystyle{<newstyle>}
% \printglossary[title=List of Terms,toctitle=Terms and abbreviations]
% BEFORE \end{document}
%%%%%%%%%%%%%%%%%%%%%%%%%%%%%%%%%%%%%%%%%%%%%
\newglossary[slg]{symbols}{syi}{syg}{Symbolverzeichnis}%
\newglossaryentry{Functional claim}%
{%
	name=Functional claim,%
	description={\dots{} represents our value proposition to our markets and customers by addressing a specific market need through the delivery of defined functions and capabilities. It reflects our expertise and responsibility in providing solutions that meet customer requirements and create differentiated value}%
}
\newglossaryentry{Functional chain}%
{%
	name=Functional chain,%
	description={\dots{} is a sequence of interconnected functions that work  together to achieve a specific operation and functionality of a system}%
}
\newglossaryentry{Product}%
{%
	name=Product,%
	description={A \dots{} is sold based on a data sheet. %
		Research and Development (R\&D) is responsible for ensuring conformity with this data sheet. %
		Additionally, the entire process follows the product lifecycle management (PLM) framework. %
		Project execution excellence refers to the ability to deliver projects that consistently meet quality, %
		time and cost objectives through well-defined processes, advanced methods and tools and a highly skilled organization. %
		It ensures that performance contracts are executed efficiently, %
		minimizing risks while maximizing value for both the customer and the organization}%
}
\newglossaryentry{Solution}%#
{%
	name=Solution,%
	description={Solutions in the Scaled Agile Framework® (SAFe®) are services, products, or systems for the customer (external or internal). %
		In SAFe®, each new development value stream develops one or more solutions. \url{https://archive.ph/pbNug}}%
}
\newglossaryentry{System}%
{%
	name=System,%
	description={A \dots{} is an organized set of components, processes and technologies that work together to perform a specific function or achieve a defined outcome. %
		They are designed to integrate seamlessly, ensuring efficiency, reliability and adaptability to meet customer or operational needs. %
		In contrast to products, a system is sold through a performance contract. %
		The business entity with PnL responsibility for the system provides a systems engineering team for ensuring compliance with the contract. %
		This process is specific to each customer contract and should follow an implementation of customer projects framework}%
}
\newglossaryentry{System of Systems}%
{%
	name=System of Systems,%
	description={A \dots {} refers to a collection of independent systems that work together to achieve a higher-level capability or objective that no single system could accomplish alone. %
		These systems maintain their own operational independence while being integrated through coordination, communication and interoperability to deliver complex mission-critical solutions}%
}
\newglossaryentry{Value Proposition}%
{%
	name=Value Proposition,%
	description={\dots {} Value proposition refers to the unique benefits we offer to address customer needs and differentiate ourselves from competitors. %
		It highlights how our products or services solve problems and deliver value effectively}%
}
\newglossaryentry{sym:pi}{%
	type=symbols,%
	name={\ensuremath{\pi}},%
	sort=pi,%
	description={Kreiszahl, Verhältnis von Umfang zu Durchmesser eines Kreises}%
}
\newacronym{aia}{AIA}{Aerospace Industries Association of America}%
\newacronym{api}{API}{Application Programming Interface}%
\newacronym{asd}{ASD}{Aerospace, Security and Defence Industries Association of Europe}%
\newacronym{cpu}{CPU}{Central Processing Unit}%
\newacronym{cra}{CRA}{Cyber Resilience Act \url{https://www.bsi.bund.de/dok/cra-en} and \url{https://digital-strategy.ec.europa.eu/en/policies/cyber-resilience-act}}%
\newacronym{cwix}{CWIX}{Coalition Warrior Interoperability Exercise}%
\newacronym{edstart}{EDSTART}{European Defence Standardisation Refence System \url{https://edstar.eda.europa.eu/}}%
\newacronym{fuewes}{FüWES}{Führungs- und Waffeneinsatz-System}%
\newacronym{gpu}{GPU}{Graphics Processing Unit}%
\newacronym{ietd}{IETD}{Interaktive Elektronische Technische Dokumentation}%
\newacronym{ilasst}{ILASST}{Integriertes Leit- und Automationssystem Schiffstechnik}%
\newacronym{mod}{MoD}{Ministry of Defence}%
\newacronym{mukdo}{MUKdo}{Marine Unterstützungskommando}%
\newacronym{öag}{öAG}{öffentlicher Auftraggeber}%
\newacronym{padm}{PAD-M}{Process Associated Document --- Mandatory}%
\newacronym{pcp}{PCP}{Product Compliance Project}%
\newacronym{pnl}{PnL}{Profit and Loss}%
\newacronym{rohs}{RoHS}{Restriction of Hazardous Substances}%
\newacronym{safe}{SAFe}{Scaled Agile Framework}%
\newacronym{sagd}{SAGD}{Schadensabwehr- und Gefechtsdienst}%
\newacronym{sbom}{SBOM}{Software Bill of Materials}%
\newacronym{sdl}{SDL}{Secure Development Lifecycle see \href{https://www.vde-verlag.de/iec-normen/249036/iec-62443-4-1-2018.html}{IEC 62443-4-1:2018}}%
\newacronym{tara}{TARA}{Threat Analysis and Risk Assessment}%
\newacronym{tavz}{TAzV}{Technische Anlage zum Vertrag}%
\newacronym{usv}{USV}{Unterbrechungsfreie Stromversorgung}%
\newacronym{wtd}{WTD}{Wehrtechnische Dienststelle}%
\newacronym{zdv}{ZDv}{Zentrale Dienstvorschrift}%

% BEFORE \begin{document}
% and print by
% \appendix
% \printindex
% \setglossarystyle{<newstyle>}
% \printglossary[title=List of Terms,toctitle=Terms and abbreviations]
% BEFORE \end{document}
%%%%%%%%%%%%%%%%%%%%%%%%%%%%%%%%%%%%%%%%%%%%%
\newglossary[slg]{symbols}{syi}{syg}{Symbolverzeichnis}%
\newglossaryentry{Functional claim}%
{%
	name=Functional claim,%
	description={\dots{} represents our value proposition to our markets and customers by addressing a specific market need through the delivery of defined functions and capabilities. It reflects our expertise and responsibility in providing solutions that meet customer requirements and create differentiated value}%
}
\newglossaryentry{Functional chain}%
{%
	name=Functional chain,%
	description={\dots{} is a sequence of interconnected functions that work  together to achieve a specific operation and functionality of a system}%
}
\newglossaryentry{Product}%
{%
	name=Product,%
	description={A \dots{} is sold based on a data sheet. %
		Research and Development (R\&D) is responsible for ensuring conformity with this data sheet. %
		Additionally, the entire process follows the product lifecycle management (PLM) framework. %
		Project execution excellence refers to the ability to deliver projects that consistently meet quality, %
		time and cost objectives through well-defined processes, advanced methods and tools and a highly skilled organization. %
		It ensures that performance contracts are executed efficiently, %
		minimizing risks while maximizing value for both the customer and the organization}%
}
\newglossaryentry{Solution}%#
{%
	name=Solution,%
	description={Solutions in the Scaled Agile Framework® (SAFe®) are services, products, or systems for the customer (external or internal). %
		In SAFe®, each new development value stream develops one or more solutions. \url{https://archive.ph/pbNug}}%
}
\newglossaryentry{System}%
{%
	name=System,%
	description={A \dots{} is an organized set of components, processes and technologies that work together to perform a specific function or achieve a defined outcome. %
		They are designed to integrate seamlessly, ensuring efficiency, reliability and adaptability to meet customer or operational needs. %
		In contrast to products, a system is sold through a performance contract. %
		The business entity with PnL responsibility for the system provides a systems engineering team for ensuring compliance with the contract. %
		This process is specific to each customer contract and should follow an implementation of customer projects framework}%
}
\newglossaryentry{System of Systems}%
{%
	name=System of Systems,%
	description={A \dots {} refers to a collection of independent systems that work together to achieve a higher-level capability or objective that no single system could accomplish alone. %
		These systems maintain their own operational independence while being integrated through coordination, communication and interoperability to deliver complex mission-critical solutions}%
}
\newglossaryentry{Value Proposition}%
{%
	name=Value Proposition,%
	description={\dots {} Value proposition refers to the unique benefits we offer to address customer needs and differentiate ourselves from competitors. %
		It highlights how our products or services solve problems and deliver value effectively}%
}
\newglossaryentry{sym:pi}{%
	type=symbols,%
	name={\ensuremath{\pi}},%
	sort=pi,%
	description={Kreiszahl, Verhältnis von Umfang zu Durchmesser eines Kreises}%
}
\newacronym{aia}{AIA}{Aerospace Industries Association of America}%
\newacronym{api}{API}{Application Programming Interface}%
\newacronym{asd}{ASD}{Aerospace, Security and Defence Industries Association of Europe}%
\newacronym{cpu}{CPU}{Central Processing Unit}%
\newacronym{cra}{CRA}{Cyber Resilience Act \url{https://www.bsi.bund.de/dok/cra-en} and \url{https://digital-strategy.ec.europa.eu/en/policies/cyber-resilience-act}}%
\newacronym{cwix}{CWIX}{Coalition Warrior Interoperability Exercise}%
\newacronym{edstart}{EDSTART}{European Defence Standardisation Refence System \url{https://edstar.eda.europa.eu/}}%
\newacronym{fuewes}{FüWES}{Führungs- und Waffeneinsatz-System}%
\newacronym{gpu}{GPU}{Graphics Processing Unit}%
\newacronym{ietd}{IETD}{Interaktive Elektronische Technische Dokumentation}%
\newacronym{ilasst}{ILASST}{Integriertes Leit- und Automationssystem Schiffstechnik}%
\newacronym{mod}{MoD}{Ministry of Defence}%
\newacronym{mukdo}{MUKdo}{Marine Unterstützungskommando}%
\newacronym{öag}{öAG}{öffentlicher Auftraggeber}%
\newacronym{padm}{PAD-M}{Process Associated Document --- Mandatory}%
\newacronym{pcp}{PCP}{Product Compliance Project}%
\newacronym{pnl}{PnL}{Profit and Loss}%
\newacronym{rohs}{RoHS}{Restriction of Hazardous Substances}%
\newacronym{safe}{SAFe}{Scaled Agile Framework}%
\newacronym{sagd}{SAGD}{Schadensabwehr- und Gefechtsdienst}%
\newacronym{sbom}{SBOM}{Software Bill of Materials}%
\newacronym{sdl}{SDL}{Secure Development Lifecycle see \href{https://www.vde-verlag.de/iec-normen/249036/iec-62443-4-1-2018.html}{IEC 62443-4-1:2018}}%
\newacronym{tara}{TARA}{Threat Analysis and Risk Assessment}%
\newacronym{tavz}{TAzV}{Technische Anlage zum Vertrag}%
\newacronym{usv}{USV}{Unterbrechungsfreie Stromversorgung}%
\newacronym{wtd}{WTD}{Wehrtechnische Dienststelle}%
\newacronym{zdv}{ZDv}{Zentrale Dienstvorschrift}%

% BEFORE \begin{document}
% and print by
% \appendix
% \printindex
% \setglossarystyle{<newstyle>}
% \printglossary[title=List of Terms,toctitle=Terms and abbreviations]
% BEFORE \end{document}
%%%%%%%%%%%%%%%%%%%%%%%%%%%%%%%%%%%%%%%%%%%%%
\newglossaryentry{Functional claim}
{
  name=Functional claim,
  description={\dots represents our value proposition to our markets and customers by addressing a specific market need through the delivery of defined functions and capabilities. It reflects our expertise and responsibility in providing solutions that meet customer requirements and create differentiated value.}
}
\newglossaryentry{Functional chain}
{
  name=Functional chain,
  description={\dots is a sequence of interconnected functions that work  together to achieve a specific operation and functionality of a system.}
}